\documentclass[12pt, a4paper]{article}

% --- Packages ---
\usepackage[utf8]{inputenc}
\usepackage[T1]{fontenc}
\usepackage[a4paper, margin=2.5cm]{geometry} % Standard margins
\usepackage{setspace} % Line spacing
\usepackage{graphicx} % For images
\usepackage{csquotes} % Recommended for biblatex
\usepackage[style=apa, backend=biber]{biblatex} % APA citations
\usepackage[acronym, toc]{glossaries} % For glossary/abbreviations
\usepackage{fancyhdr} % For custom headers and footers
\usepackage[hidelinks]{hyperref} % Clickable links in TOC and citations

% --- Configuration ---
\addbibresource{literature.bib} % Link bibliography file
\onehalfspacing % 1.5 Line spacing (common for seminar papers)
\setlength{\headheight}{15pt} % Fix fancyhdr warning
\makeglossaries % Generate the glossary

% --- Header and Footer Setup ---
\pagestyle{fancy}
\fancyhf{} % Clear all header and footer fields
\renewcommand{\headrulewidth}{0.4pt} % Thin line under header
\renewcommand{\footrulewidth}{0.4pt} % Thin line over footer

% Header
\fancyhead[R]{\thepage} % Page number on top right

% Footer
\fancyfoot[L]{\small Gianluca Decaro} % Name on left
\fancyfoot[R]{\small UXDM\_CUR} % Course ID on right

% --- Glossary Entries (Example) ---
\newacronym{ux}{UX}{User Experience}
\newacronym{ai}{AI}{Artificial Intelligence}

\begin{document}

% --- Title Page ---
\begin{titlepage}
    \centering
    % Logo
    \includegraphics[width=0.5\textwidth]{images/thiRGB.jpg}
    \vspace{2cm}
    
    % Course Name
    {\scshape\Large Critical UX Reflections - UXDM\_CUR \par}
    \vspace{1.5cm}
    
    % Title
    {\huge\bfseries Cultural Appropriation in UX Design \par}
    \vspace{2cm}
    
    % Author
    {\Large\bfseries Gianluca Decaro \par}
    {\large Matrikelnummer: 00121853 \par}
    \vfill
    
    % Professor and Date
    \large
    Professor: Prof. Dr. Laura Crompton\par
    \vspace{0.5cm}
    \today
\end{titlepage}

% --- AI Disclaimer ---
\thispagestyle{empty} % No page number on this page
\vspace*{5cm} % Vertical spacing to center it visually
\section*{Note on the Use of Artificial Intelligence}
In this work, artificial intelligence tools were used solely for grammatical error correction and stylistic refinement. These adjustments were made carefully, considering the original content, and aim to optimize the quality and comprehensibility of this work.
\newpage

% --- Table of Contents ---
\pagenumbering{roman} % Front matter uses roman numerals (i, ii...)
\tableofcontents
\newpage

% --- Main Content ---
\pagenumbering{arabic} % Main content starts at page 1
\section{Introduction}
Imagine Tom, a design student working on his first major project: a meditation app he calls ``Zenify.'' To make the app feel ``authentic,'' Tom searches online for inspiration. He finds a silhouette of a Buddha for the logo, selects a ``Japanese Buddhist Pattern'' for the background to add some visual flair, and designs a gamification feature where users earn ``Karma Points''. Tom is proud of his work; he believes he has created a beautiful, spiritually resonant product. However, Tom has likely engaged in cultural appropriation. He has taken symbols of deep religious and cultural significance---Buddhist iconography, the concept of Karma, the Zen tradition---and stripped them of their original context to serve a commercial purpose, likely without understanding the history or meaning behind them.

Tom's case is fictional, yet it reflects a common reality in the field of User Experience (UX) Design. In an era defined by globalization and the internet, access to cultural artifacts and aesthetics is unprecedentedly easy. As \textcite{arya_cultural_2021} argues, this ease has made cultural boundaries ``precarious'' and rendered the issue of cultural appropriation an urgent ethical concern. In the fast-paced environment of digital product design, where aesthetics are often used to ``spice'' up interfaces \parencite{arya_cultural_2021}, designers often unknowingly reinforce systemic oppression by treating culture as a mere resource to be mined. This raises fundamental questions about the \textit{agency} and \textit{autonomy} of designers: are they merely following market trends, or can they act as responsible ethical agents who reflect on the normative implications of their design decisions?

The central problem is that while cultural exchange is a rich part of human experience, specific forms of ``taking'' in design can lead to commodification and the erasure of meaning. This paper addresses the following research question: \textit{How can user experience (UX) designers ethically navigate the tension between cross-cultural inspiration and cultural appropriation to avoid perpetuating systemic oppression and commodification?}

This paper argues that while cultural inspiration is a vital part of creativity, without a critical understanding of power dynamics and context, it risks becoming harmful appropriation. To act \textit{morally}---that is, to align design practice with principles of justice and respect---UX designers must move beyond superficial ``borrowing''. Instead, they should adopt a \textit{hermeneutic process of understanding} \parencite{schneider_appropriation_2003} and embrace \textit{participatory design} methods \parencite{vasalou_problematizing_2014} to ensure ethical engagement.

The paper is structured as follows: First, it defines cultural appropriation and the challenges of defining ``culture'' itself. Second, it establishes an ethical framework based on the ``oppression account'' \parencite{matthes_cultural_2019} and concepts of commodification. Third, it examines how these issues manifest specifically in UX practice. Finally, it proposes a practical framework---the ``3 Ps'' of Power, Profit, and Participation---validated by reflections from a discussion with UX design students, to guide designers from appropriation toward appreciation.

\section{Defining the Terrain: Culture, Appropriation, and the ``Fuzzy'' Box}

To understand the ethical implications of Tom's design choices, we must first define the terrain. The term ``appropriation'' stems from the Latin \textit{appropriare}, meaning ``to make one's own'' \parencite{schneider_appropriation_2003}. In the academic context, \textcite{arya_cultural_2021}, drawing on \textcite{young_cultural_2010}, defines cultural appropriation as ``the taking of items (whether tangible or intangible) including ideas, from one culture by another.'' Similarly, \textcite{ziff_borrowed_1997} emphasize the unauthorized nature of this act, describing it as taking intellectual property, cultural expressions, or ways of knowledge from a culture that is not one's own.

However, not every act of taking constitutes harmful appropriation. \textcite{lenard_what_2020} provide a rigorous framework for identifying cultural appropriation through four necessary conditions. First, the \textbf{Taking Condition}: there must be an act of using a symbol or practice that did not originate with the user. This distinguishes appropriation from mere influence. Second, the \textbf{Value Condition}: the item must hold significant meaning for the originating culture. It is not enough for an item to be merely associated with a culture; it must be ordinarily central to the culture's collective life or recognizable by its members as a marker of identity. Third, the \textbf{Contested Context Condition}: members of that culture must voice objection to the use, indicating a shared sense of harm or violation. \textcite{lenard_what_2020} specify that this contestation must be meaningful---sustained over time and voiced by multiple members---rather than a fleeting complaint. Finally, the \textbf{Knowledge or Culpable Ignorance Condition}: the appropriator either knows the value and contestation or \textit{should have known}. In the digital age, where information is readily available, ignorance is rarely a valid defense. Tom, for instance, could have easily researched the religious significance of the Buddha image; his failure to do so constitutes culpable ignorance.

A significant challenge in defining appropriation is the concept of ``culture'' itself. To say someone appropriated from a culture assumes that cultures are fixed entities---separate ``boxes'' with clear boundaries between insiders and outsiders. \textcite{arya_cultural_2021} challenges this view, arguing that culture is ``amorphous'', fluid, and constantly changing. Identities are overlapping; a person might be an insider ethnically but an outsider geographically.

\textcite{arya_cultural_2021} illustrates this complexity with the example of hip-hop. Originating with African-Americans in the South Bronx as an expression of resistance against racism, hip-hop seems to have a clear ``insider'' group. However, the genre has become a global phenomenon, adapted by disaffected ethnic minority youth in Europe to voice their \textit{own} local struggles. Are these European artists ``appropriating'' black culture, or are they participating in a shared ethos of resistance? This example demonstrates that cultural boundaries are porous. In UX, this problem manifests when designers rely on rigid user personas like ``The Millennial User,'' assuming a unified cultural framework that ignores the intersectional realities of class, geography, and ethnicity. While boundaries are precarious, the power dynamics between dominant and marginalized groups remain real, making the ethical question not just about \textit{who} owns culture, but \textit{who} has the power to profit from it.

Finally, it is crucial to distinguish cultural appropriation from other forms of cultural wrong. \textcite{lenard_what_2020} identify \textbf{Cultural Offense}, which involves acts that insult or humiliate a group (e.g., a racist caricature), and \textbf{Cultural Misrepresentation}, which involves stereotyping or creating a false depiction (e.g., reducing Mexican culture to tacos and sombreros). While these often overlap with appropriation, appropriation specifically involves the taking of something of \textit{value} and repurposing it, often leading to the specific harms of commodification and exploitation discussed in the next section.

\section{The Ethics of ``Taking'': Harm, Power, and Commodification}

Having defined the conditions of appropriation, we must ask: why is it morally wrong? \textcite{matthes_cultural_2019} provides the ``oppression account'' to answer this. He argues that the wrongness of cultural appropriation is not merely about hurt feelings or offense; rather, it is rooted in systemic imbalances of power. When a dominant culture takes from a marginalized one, the act is not neutral. It interacts with and exacerbates existing structures of oppression. In this view, appropriation is a political act that reinforces the marginalization of the source group. \textcite{matthes_cultural_2019} extends this by discussing how appropriation can function as a form of ``epistemic violence,'' where the dominant group silences or speaks \textit{for} the marginalized group, denying them the credibility to define their own experience.

A primary mechanism of this harm is ``commodification,'' which \textcite{arya_cultural_2021} describes as the process of turning a cultural good into an object of trade. In this transformation, the item's ``Use Value''---its spiritual, social, or community meaning---is stripped away and replaced by ``Exchange Value'', or its commercial worth. \textcite{schneider_appropriation_2003} refers to this as ``stripping context'', where a symbol is removed from its original web of meaning to serve a foreign purpose.

This process can lead to a phenomenon \textcite{arya_cultural_2021} calls the ``pizza effect'' or ``re-enculturation''. This occurs when a dominant culture appropriates a practice, repackages it, and sells it back to the originating culture, often displacing the original tradition. A pertinent example for UX design is the global wellness industry. Yoga and mindfulness, deeply rooted in Indian philosophy, have been repackaged by Western tech companies into sleek, gamified apps. These products are then marketed globally, including back to India, where the Westernized, commodified version often gains prestige over traditional practices. This cycle of harm reinforces cultural imperialism, where the dominant culture claims the power to define what is ``modern'' or ``valuable.''

However, this ethical stance is not universally accepted. \textcite{lenard_what_2020} note a ``sceptical view'' which argues that cultures have always borrowed from one another and that such exchange is natural. A more nuanced perspective is the ``Respectful Exchange View'' proposed by \textcite{young_cultural_2010}. Young argues that appropriation is not \textit{inherently} wrong if the appropriator provides proper acknowledgment and treats the material with respect. He warns that a strict prohibition on borrowing could stifle creativity.

Taking this argument further, the novelist \textcite{shriver_lionel_2016} champions the ``Artistic Freedom View''. In a controversial 2016 speech, Shriver donned a sombrero to mock what she saw as the absurdity of identity politics. She argued that the demand to ``keep our mitts off experience that doesn't belong to us'' threatens the very soul of fiction. For Shriver, the writer's job is to imagine themselves into other lives; restricting this right leads to ``anodyne'' (bland) art where creators are paralyzed by the fear of offending. She dismisses the concept of appropriation as a ``passing fad'' and a form of ``surveillance'' that polices imagination.

While Shriver's defense of the imagination is compelling in the realm of literature, it falters when applied to User Experience Design. The distinction lies in the purpose of the work. Fiction asks a reader to suspend disbelief for an emotional journey; UX asks a user to perform a task within a commercial ecosystem. When a designer like Tom uses a religious icon to sell a subscription service, he is not engaging in a deep, empathetic exercise of ``stepping into another's shoes,'' as Shriver advocates. He is stripping a symbol of its context to generate profit. In this functional, commercial environment, the risk is not just ``bland art,'' but active harm. By reinforcing stereotypes or commodifying sacred traditions, UX designers contribute to the very systemic oppression that \textcite{matthes_cultural_2019} warns against. Therefore, in the context of design ethics, the ``oppression account'' must take precedence over the ``artistic freedom'' defense.

\section{Cultural Appropriation in UX Practice}

The theoretical harms of appropriation are not abstract concepts in UX; they manifest daily in design workflows. One of the most prevalent instances occurs in Visual and UI Design through what \textcite{arya_cultural_2021}, citing bell hooks, describes as the ``spice'' analogy. In this dynamic, elements of marginalized cultures are used as ``seasoning'' to liven up mainstream digital products. A designer might use a sacred Indigenous pattern or a Hindu deity icon simply to create an ``earthly'' or ``authentic'' vibe for a lifestyle app. By treating these symbols as mere aesthetic assets, the designer strips them of their historical and social weight, reinforcing the ``asymmetry of power'' where the dominant culture defines what is ``cool'' without bearing the responsibility of understanding the source \parencite{arya_cultural_2021}.

Beyond the interface, cultural appropriation often hides within the methodologies of UX research, specifically in the creation of user personas. Standard practice involves creating neat demographic profiles, such as ``The Gen Z User'' or ``The Hispanic Market''. However, these models risk becoming ``essentialist boxes'' that flatten the complexities of human identity. As \textcite{arya_cultural_2021} points out, culture is not a fixed category but a fluid one shaped by ``interlocking social categorisations'' like race, gender, and class. By relying on rigid, stereotypical personas, designers ignore the \textit{intersectionality} of their users. This not only leads to hollow design but can actively cause \textit{cultural offense} by reproducing harmful tropes that users must then navigate.

A complex dimension of this issue is the tension between the designer's intent and the user's interpretation. \textcite{vasalou_problematizing_2014} explore this in their study of a serious game inspired by \textit{Día de los Muertos}, a Mexican festival celebrating the dead. The researchers aimed for cultural authenticity by highlighting core values like remembrance and acceptance. However, they discovered that even with an authentic narrative, children began to appropriate the cultural rituals, filtering them through their own religious and media-influenced beliefs about death.

Specifically, the study found that children often reproduced the festival's themes in shallow or distorted ways. Some children constructed narratives where death primarily affected ``people of status,'' contradicting the festival's message of universality \parencite{vasalou_problematizing_2014}. More troublingly, some teams introduced counterproductive ways of coping with bereavement, such as narratives of violent revenge where characters ``casually killed'' others, or scenarios where characters were punished for expressing grief. 

This case study reveals a critical UX challenge: the ``clash'' between the authentic cultural message and the user's existing, potentially problematic, beliefs. It suggests that the designer's ethical responsibility is not limited to the \textit{source culture} being represented but extends to the \textit{end user}. Designers must be cognizant of how their products might encourage shallow cultural interpretations or reinforce detrimental beliefs in their audience. This expanded obligation requires a move toward a design process that actively manages these cultural interpretations rather than simply applying a superficial cultural ``skin'' to a product.

\section{From Appropriation to Appreciation: A Framework for Designers}

If cultural appropriation is the problem, what is the solution? The goal is to move from appropriation to \textit{cultural appreciation}. \textcite{han_moving_2019} and \textcite{arya_cultural_2021} help distinguish the two: while appropriation takes a ``snapshot'' of a culture and strips it of its social history for commercial gain, appreciation engages with the full context. Appreciation prioritizes the ``Use Value''---the spiritual or community meaning---over the ``Exchange Value.'' Most importantly, appreciation is built on dialogue and exchange rather than unauthorized taking.

To help designers navigate this shift in a practical setting, I propose the ``3 Ps Framework,'' adapted from \textcite{tedx_talks_3_2022}. This diagnostic tool allows designers to evaluate their work before implementation:

\begin{enumerate}
    \item \textbf{Power:} The first step is to check the context. Designers must ask: ``Am I part of a dominant culture taking from a marginalized one?'' If the answer is yes, the power dynamic is inherently unequal, and the risk of reinforcing oppression is high \parencite{matthes_cultural_2019}. This does not forbid inspiration, but it demands extreme caution and responsibility.
    \item \textbf{Profit:} The second step is to check the motive. ``Am I using this cultural element simply to add aesthetic `spice' or to sell a product?'' If a sacred symbol is being commodified for financial gain without benefiting the source community, it is harmful appropriation.
    \item \textbf{Participation:} The final step is to check the agency. ``Who is involved in this design process?'' As \textcite{han_moving_2019} suggests, appreciation requires participation and permission. If the source community is not involved, the design is likely an unauthorized taking.
\end{enumerate}

To truly satisfy the condition of ``Participation,'' designers must go beyond simple consultation and embrace \textit{Participatory Design}. As \textcite{vasalou_problematizing_2014} suggest, co-designing with cultural insiders can help mitigate the risks of misrepresentation. However, this requires a fundamental shift in the designer's mindset, which \textcite{schneider_appropriation_2003} describes as a ``hermeneutic procedure.'' This is not merely a method but a deep process of interpretation that moves beyond the superficial ``taking'' of objects.

Schneider breaks this procedure into a cycle of understanding that challenges the appropriator. The first and most critical step is \textbf{relinquishment}. The designer must ``let go'' of their own ego, habits, and pre-existing assumptions. This is difficult in a commercial context where the designer is often positioned as the expert solver of problems. Relinquishment requires admitting ignorance and stepping back to allow the ``other'' to speak.

The second step is \textbf{engagement} with the ``alien'' or ``other.'' This is not about cataloging exotic traits for a mood board but about engaging with the meaning of the culture on its own terms. Finally, this process leads to \textbf{transformation}. In a true hermeneutic exchange, the designer does not just acquire a new asset; they acquire a ``new self-understanding'' \parencite{schneider_appropriation_2003}. They are changed by the encounter.

This philosophical stance provides the ethical foundation for the practical methodology of Participatory Design. You cannot truly co-design if you have not first relinquished the ego of the solitary author. By adopting this hermeneutic approach, designers can transform themselves from appropriators into responsible mediators who facilitate genuine cultural exchange rather than exploitation.

\section{Reflection on Interactive Class Activity: Validating the Framework}

To test these theoretical frameworks against the reality of design practice, I conducted an interactive discussion with my peers in the ``Critical UX Reflections'' course. The session aimed to explore whether the tension between artistic freedom and ethical responsibility is felt by emerging designers.

The discussion strongly challenged the ``Artistic Freedom'' view held by Lionel Shriver. When asked if avoiding cultural references limits creativity, one student provided a powerful rebuttal: ``If you can't design good experiences or a good product without appropriating or harming another culture, are you really a good designer?'' The class consensus was that relying on cultural clichés---such as ``stamping sombreros on anything remotely Mexican''---represents a \textit{lack} of creativity, not the freedom of it. This practitioner's perspective validates the argument that ethical constraints do not stifle imagination; rather, they demand higher quality, less lazy design work.

Furthermore, the discussion reinforced the relevance of \textcite{matthes_cultural_2019}'s ``oppression account.'' Students argued that ``you're not celebrating a culture if you are refusing to hear a culture'', explicitly acknowledging the role of the ``oppressor'' culture. This aligns with \textcite{lenard_what_2020}'s ``Contested Context Condition'', confirming that designers are acutely aware that ignoring the voices of the marginalized is a structural harm, not just a faux pas.

Finally, the peers intuitively arrived at the same solution proposed in this paper's framework. One student noted that ``just some interviews in the explorative phase... may not be enough'' and suggested ``bringing the people on board while developing''. This independent conclusion---that designers must move from solitary creators to collaborators---empirically validates the necessity of the \textit{Participatory Design} approach advocated by \textcite{vasalou_problematizing_2014}. It demonstrates that for the next generation of designers, the move from appropriation to appreciation is not just an academic ideal, but a practical requirement for high-quality work.

\section{Conclusion}

The story of Tom and his ``Zenify'' app serves as a cautionary tale for the modern UX designer. It illustrates that cultural appropriation is rarely an act of malicious theft; often, it is an act of unthinking ``taking'' driven by commercial urgency and a lack of critical reflection. However, as this paper has argued, the absence of bad intent does not mitigate the harm. When designers like Tom strip cultural symbols of their context and commodify them for profit, they participate in a political act that reinforces systemic oppression \parencite{matthes_cultural_2019}.

Returning to the research question---how UX designers can ethically navigate the tension between inspiration and appropriation---the answer lies in a fundamental shift in practice. Designers must move from a model of \textit{aesthetic borrowing}, where culture is treated as a resource to be mined, to a model of \textit{hermeneutic understanding} \parencite{schneider_appropriation_2003}. This requires relinquishing the ego of the solitary creator and embracing the humility of the collaborator.

Frameworks like the ``3 Ps'' (Power, Profit, Participation) provide the necessary guardrails for this journey, but the ultimate solution is structural. To avoid the ``pizza effect'' of re-enculturating marginalized groups through Western technology \parencite{arya_cultural_2021}, designers must actively involve those groups in the creation process. By adopting participatory design methods, UX professionals can transform themselves from appropriators into responsible ``interfaces'' who mediate between cultures with respect, ensuring that digital products celebrate human diversity rather than exploiting it.

% --- Bibliography ---
\newpage
\pagenumbering{Roman} % Back matter uses Uppercase Roman (I, II...)
\printbibliography

% --- List of Figures & Glossary ---
\newpage
\listoffigures
\newpage
\printglossary[type=\acronymtype, title={Glossary}]

% --- Appendix ---
\newpage
\appendix
\section{Appendix Title}
Any supplementary material (interviews, large tables, etc.) goes here.

\end{document}
