\documentclass[12pt, a4paper]{article}

% --- Packages ---
\usepackage[utf8]{inputenc}
\usepackage[T1]{fontenc}
\usepackage[a4paper, margin=2.5cm]{geometry} % Standard margins
\usepackage{setspace} % Line spacing
\usepackage{graphicx} % For images
\usepackage{csquotes} % Recommended for biblatex
\usepackage[style=apa, backend=biber]{biblatex} % APA citations
\usepackage[acronym, toc]{glossaries} % For glossary/abbreviations
\usepackage{fancyhdr} % For custom headers and footers
\usepackage[hidelinks]{hyperref} % Clickable links in TOC and citations

% --- Configuration ---
\addbibresource{literature.bib} % Link bibliography file
\onehalfspacing % 1.5 Line spacing (common for seminar papers)
\setlength{\headheight}{15pt} % Fix fancyhdr warning
\makeglossaries % Generate the glossary

% --- Header and Footer Setup ---
\pagestyle{fancy}
\fancyhf{} % Clear all header and footer fields
\renewcommand{\headrulewidth}{0.4pt} % Thin line under header
\renewcommand{\footrulewidth}{0.4pt} % Thin line over footer

% Header
\fancyhead[R]{\thepage} % Page number on top right

% Footer
\fancyfoot[L]{\small Gianluca Decaro} % Name on left
\fancyfoot[R]{\small UXDM\_CUR} % Course ID on right

% --- Glossary Entries (Example) ---
\newacronym{ux}{UX}{User Experience}
\newacronym{ai}{AI}{Artificial Intelligence}

\begin{document}

% --- Title Page ---
\begin{titlepage}
    \centering
    % Logo
    \includegraphics[width=0.5\textwidth]{images/thiRGB.jpg}
    \vspace{2cm}
    
    % Course Name
    {\scshape\Large Critical UX Reflections - UXDM\_CUR \par}
    \vspace{1.5cm}
    
    % Title
    {\huge\bfseries Cultural Appropriation in UX Design \par}
    \vspace{2cm}
    
    % Author
    {\Large\bfseries Gianluca Decaro \par}
    {\large Matrikelnummer: 00121853 \par}
    \vfill
    
    % Professor and Date
    \large
    Professor: Prof. Dr. Laura Crompton\par
    \vspace{0.5cm}
    \today
\end{titlepage}

% --- AI Disclaimer ---
\thispagestyle{empty} % No page number on this page
\vspace*{5cm} % Vertical spacing to center it visually
\section*{Note on the Use of Artificial Intelligence}
In this work, artificial intelligence tools were used solely for grammatical error correction and stylistic refinement. These adjustments were made carefully, considering the original content, and aim to optimize the quality and comprehensibility of this work.
\newpage

% --- Table of Contents ---
\pagenumbering{roman} % Front matter uses roman numerals (i, ii...)
\tableofcontents
\newpage

% --- Main Content ---
\pagenumbering{arabic} % Main content starts at page 1
\section{Introduction}
Imagine Tom, a design student working on his first major project: a meditation app he calls ``Zenify.'' To make the app feel ``authentic,'' Tom searches online for inspiration. He finds a silhouette of a Buddha for the logo, selects a ``Japanese Buddhist Pattern'' for the background to add some visual flair, and designs a gamification feature where users earn ``Karma Points''. Tom is proud of his work; he believes he has created a beautiful, spiritually resonant product. However, Tom has likely engaged in cultural appropriation. He has taken symbols of deep religious and cultural significance---Buddhist iconography, the concept of Karma, the Zen tradition---and stripped them of their original context to serve a commercial purpose, likely without understanding the history or meaning behind them.

Tom's case is fictional, yet it reflects a common reality in the field of User Experience (UX) Design. In an era defined by globalization and the internet, access to cultural artifacts and aesthetics is unprecedentedly easy. As \textcite{arya_cultural_2021} argues, this ease has made cultural boundaries ``precarious'' and rendered the issue of cultural appropriation an urgent ethical concern. In the fast-paced environment of digital product design, where aesthetics are often used to ``spice'' up interfaces \parencite{arya_cultural_2021}, designers often unknowingly reinforce systemic oppression by treating culture as a mere resource to be mined. This raises fundamental questions about the \textit{agency} and \textit{autonomy} of designers: are they merely following market trends, or can they act as responsible ethical agents who reflect on the normative implications of their design decisions?

The central problem is that while cultural exchange is a rich part of human experience, specific forms of ``taking'' in design can lead to commodification and the erasure of meaning. This paper addresses the following research question: \textit{How can user experience (UX) designers ethically navigate the tension between cross-cultural inspiration and cultural appropriation to avoid perpetuating systemic oppression and commodification?}

This paper argues that while cultural inspiration is a vital part of creativity, without a critical understanding of power dynamics and context, it risks becoming harmful appropriation. To act \textit{morally}---that is, to align design practice with principles of justice and respect---UX designers must move beyond superficial ``borrowing''. Instead, they should adopt a \textit{hermeneutic process of understanding} \parencite{schneider_appropriation_2003} and embrace \textit{participatory design} methods \parencite{vasalou_problematizing_2014} to ensure ethical engagement.

The paper is structured as follows: First, it defines cultural appropriation and the challenges of defining ``culture'' itself. Second, it establishes an ethical framework based on the ``oppression account'' \parencite{matthes_cultural_2019} and concepts of commodification. Third, it examines how these issues manifest specifically in UX practice. Finally, it proposes a practical framework---the ``3 Ps'' of Power, Profit, and Participation---validated by reflections from a discussion with UX design students, to guide designers from appropriation toward appreciation.

\section{Analysis}
Your main analysis...

\section{Conclusion}
Summary of findings...

% --- Bibliography ---
\newpage
\pagenumbering{Roman} % Back matter uses Uppercase Roman (I, II...)
\printbibliography

% --- List of Figures & Glossary ---
\newpage
\listoffigures
\newpage
\printglossary[type=\acronymtype, title={Glossary}]

% --- Appendix ---
\newpage
\appendix
\section{Appendix Title}
Any supplementary material (interviews, large tables, etc.) goes here.

\end{document}
