\documentclass[12pt, a4paper]{article}

% --- Packages ---
\usepackage[utf8]{inputenc}
\usepackage[T1]{fontenc}
\usepackage[a4paper, margin=2.5cm]{geometry} % Standard margins
\usepackage{setspace} % Line spacing
\usepackage{graphicx} % For images
\usepackage{csquotes} % Recommended for biblatex
\usepackage[style=apa, backend=biber]{biblatex} % APA citations
\usepackage[acronym, toc]{glossaries} % For glossary/abbreviations
\usepackage{fancyhdr} % For custom headers and footers
\usepackage[hidelinks]{hyperref} % Clickable links in TOC and citations

% --- Configuration ---
\addbibresource{literature.bib} % Link bibliography file
\onehalfspacing % 1.5 Line spacing (common for seminar papers)
\makeglossaries % Generate the glossary

% --- Header and Footer Setup ---
\pagestyle{fancy}
\fancyhf{} % Clear all header and footer fields
\renewcommand{\headrulewidth}{0.4pt} % Thin line under header
\renewcommand{\footrulewidth}{0.4pt} % Thin line over footer

% Header
\fancyhead[R]{\thepage} % Page number on top right

% Footer
\fancyfoot[L]{\small Gianluca Decaro} % Name on left
\fancyfoot[R]{\small UXDM\_CUR} % Course ID on right

% --- Glossary Entries (Example) ---
\newacronym{ux}{UX}{User Experience}
\newacronym{ai}{AI}{Artificial Intelligence}

\begin{document}

% --- Title Page ---
\begin{titlepage}
    \centering
    % Logo
    \includegraphics[width=0.6\textwidth]{images/thiRGB.jpg}
    \vspace{2cm}
    
    % Course Name
    {\scshape\Large Critical UX Reflections - UXDM\_CUR \par}
    \vspace{1.5cm}
    
    % Title
    {\huge\bfseries Cultural Appropriation in UX Design \par}
    \vspace{2cm}
    
    % Author
    {\Large\bfseries Gianluca Decaro \par}
    {\large Matrikelnummer: 00121853 \par}
    \vfill
    
    % Professor and Date
    \large
    Professor: Prof. Dr. Laura Crompton\par
    \vspace{0.5cm}
    \today
\end{titlepage}

% --- AI Disclaimer ---
\thispagestyle{empty} % No page number on this page
\vspace*{5cm} % Vertical spacing to center it visually
\section*{Note on the Use of Artificial Intelligence}
In this work, artificial intelligence tools were used solely for grammatical error correction and stylistic refinement. These adjustments were made carefully, considering the original content, and aim to optimize the quality and comprehensibility of this work.
\newpage

% --- Table of Contents ---
\tableofcontents
\newpage

% --- Main Content ---
\section{Introduction}
Your introduction goes here. You can cite sources like this \parencite{dummy2026}.

\section{Analysis}
Your main analysis...

\section{Conclusion}
Summary of findings...

% --- Bibliography ---
\newpage
\printbibliography

% --- List of Figures & Glossary ---
\newpage
\listoffigures
\newpage
\printglossary[type=\acronymtype, title={List of Abbreviations}]

% --- Appendix ---
\newpage
\appendix
\section{Appendix Title}
Any supplementary material (interviews, large tables, etc.) goes here.

\end{document}
