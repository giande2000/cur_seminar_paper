\documentclass[12pt, a4paper]{article}

% --- Packages ---
\usepackage[utf8]{inputenc}
\usepackage[T1]{fontenc}
\usepackage[a4paper, margin=2.5cm]{geometry} % Standard margins
\usepackage{setspace} % Line spacing
\usepackage{graphicx} % For images
\usepackage{csquotes} % Recommended for biblatex
\usepackage[style=apa, backend=biber]{biblatex} % APA citations
\usepackage[acronym, toc]{glossaries} % For glossary/abbreviations
\usepackage{fancyhdr} % For custom headers and footers
\usepackage[hidelinks]{hyperref} % Clickable links in TOC and citations

% --- Configuration ---
\addbibresource{literature.bib} % Link bibliography file
\onehalfspacing % 1.5 Line spacing (common for seminar papers)
\setlength{\headheight}{15pt} % Fix fancyhdr warning
\makeglossaries % Generate the glossary

% --- Header and Footer Setup ---
\pagestyle{fancy}
\fancyhf{} % Clear all header and footer fields
\renewcommand{\headrulewidth}{0.4pt} % Thin line under header
\renewcommand{\footrulewidth}{0.4pt} % Thin line over footer

% Header
\fancyhead[R]{\thepage} % Page number on top right

% Footer
\fancyfoot[L]{\small Gianluca Decaro} % Name on left
\fancyfoot[R]{\small UXDM\_CUR} % Course ID on right

% --- Glossary Entries (Example) ---
\newacronym{ux}{UX}{User Experience}
\newacronym{ai}{AI}{Artificial Intelligence}

\begin{document}

% --- Title Page ---
\begin{titlepage}
    \centering
    % Logo
    \includegraphics[width=0.5\textwidth]{images/thiRGB.jpg}
    \vspace{2cm}
    
    % Course Name
    {\scshape\Large Critical UX Reflections - UXDM\_CUR \par}
    \vspace{1.5cm}
    
    % Title
    {\huge\bfseries Cultural Appropriation in UX Design \par}
    \vspace{2cm}
    
    % Author
    {\Large\bfseries Gianluca Decaro \par}
    {\large Matrikelnummer: 00121853 \par}
    \vfill
    
    % Professor and Date
    \large
    Professor: Prof. Dr. Laura Crompton\par
    \vspace{0.5cm}
    \today
\end{titlepage}

% --- AI Disclaimer ---
\thispagestyle{empty} % No page number on this page
\vspace*{5cm} % Vertical spacing to center it visually
\section*{Note on the Use of Artificial Intelligence}
In this work, artificial intelligence tools were used solely for grammatical error correction and stylistic refinement. These adjustments were made carefully, considering the original content, and aim to optimize the quality and comprehensibility of this work.
\newpage

% --- Table of Contents ---
\pagenumbering{roman} % Front matter uses roman numerals (i, ii...)
\tableofcontents
\newpage

% --- Main Content ---
\pagenumbering{arabic} % Main content starts at page 1
\section{Introduction}
Imagine Tom, a design student working on his first major project: a meditation app he calls ``Zenify.'' To make the app feel ``authentic,'' Tom searches online for inspiration. He finds a silhouette of a Buddha for the logo, selects a ``Japanese Buddhist Pattern'' for the background to add some visual flair, and designs a gamification feature where users earn ``Karma Points''. Tom is proud of his work; he believes he has created a beautiful, spiritually resonant product. However, Tom has likely engaged in cultural appropriation. He has taken symbols of deep religious and cultural significance---Buddhist iconography, the concept of Karma, the Zen tradition---and stripped them of their original context to serve a commercial purpose, likely without understanding the history or meaning behind them.

Tom's case is fictional, yet it reflects a common reality in the field of User Experience (UX) Design. In an era defined by globalization and the internet, access to cultural artifacts and aesthetics is unprecedentedly easy. As \textcite{arya_cultural_2021} argues, this ease has made cultural boundaries ``precarious'' and rendered the issue of cultural appropriation an urgent ethical concern. In the fast-paced environment of digital product design, where aesthetics are often used to ``spice'' up interfaces \parencite{arya_cultural_2021}, designers often unknowingly reinforce systemic oppression by treating culture as a mere resource to be mined. This raises fundamental questions about the \textit{agency} and \textit{autonomy} of designers: are they merely following market trends, or can they act as responsible ethical agents who reflect on the normative implications of their design decisions?

The central problem is that while cultural exchange is a rich part of human experience, specific forms of ``taking'' in design can lead to commodification and the erasure of meaning. This paper addresses the following research question: \textit{How can user experience (UX) designers ethically navigate the tension between cross-cultural inspiration and cultural appropriation to avoid perpetuating systemic oppression and commodification?}

This paper argues that while cultural inspiration is a vital part of creativity, without a critical understanding of power dynamics and context, it risks becoming harmful appropriation. To act \textit{morally}---that is, to align design practice with principles of justice and respect---UX designers must move beyond superficial ``borrowing''. Instead, they should adopt a \textit{hermeneutic process of understanding} \parencite{schneider_appropriation_2003} and embrace \textit{participatory design} methods \parencite{vasalou_problematizing_2014} to ensure ethical engagement.

The paper is structured as follows: First, it defines cultural appropriation and the challenges of defining ``culture'' itself. Second, it establishes an ethical framework based on the ``oppression account'' \parencite{matthes_cultural_2019} and concepts of commodification. Third, it examines how these issues manifest specifically in UX practice. Finally, it proposes a practical framework---the ``3 Ps'' of Power, Profit, and Participation---validated by reflections from a discussion with UX design students, to guide designers from appropriation toward appreciation.

\section{Defining the Terrain: Culture, Appropriation, and the ``Fuzzy'' Box}

To understand the ethical implications of Tom's design choices, we must first define the terrain. The term ``appropriation'' stems from the Latin \textit{appropriare}, meaning ``to make one's own'' \parencite{schneider_appropriation_2003}. In the academic context, \textcite{arya_cultural_2021}, drawing on \textcite{young_cultural_2010}, defines cultural appropriation as ``the taking of items (whether tangible or intangible) including ideas, from one culture by another.'' Similarly, \textcite{ziff_borrowed_1997} emphasize the unauthorized nature of this act, describing it as taking intellectual property, cultural expressions, or ways of knowledge from a culture that is not one's own.

However, not every act of taking constitutes harmful appropriation. \textcite{lenard_what_2020} provide a rigorous framework for identifying cultural appropriation through four necessary conditions. First, the \textbf{Taking Condition}: there must be an act of using a symbol or practice that did not originate with the user. Second, the \textbf{Value Condition}: the item must hold significant meaning for the originating culture; it is not merely a random artifact. Third, the \textbf{Contested Context Condition}: members of that culture must voice objection to the use, indicating a shared sense of harm or violation. Finally, the \textbf{Knowledge or Culpable Ignorance Condition}: the appropriator either knows the value and contestation or \textit{should have known}. In the digital age, where information is readily available, ignorance is rarely a valid defense. Tom, for instance, could have easily researched the religious significance of the Buddha image; his failure to do so constitutes culpable ignorance.

A significant challenge in defining appropriation is the concept of ``culture'' itself. To say someone appropriated from a culture assumes that cultures are fixed entities---separate ``boxes'' with clear boundaries between insiders and outsiders. \textcite{arya_cultural_2021} challenges this view, arguing that culture is ``amorphous'', fluid, and constantly changing. Identities are overlapping; a person might be an insider ethnically but an outsider geographically. This ``fuzziness'' makes the issue precarious but does not dismiss it. While boundaries are porous, the power dynamics between dominant and marginalized groups remain real.

Finally, it is crucial to distinguish cultural appropriation from other forms of cultural wrong. \textcite{lenard_what_2020} identify \textbf{Cultural Offense}, which involves acts that insult or humiliate a group (e.g., a racist caricature), and \textbf{Cultural Misrepresentation}, which involves stereotyping or creating a false depiction (e.g., reducing Mexican culture to tacos and sombreros). While these often overlap with appropriation, appropriation specifically involves the taking of something of \textit{value} and repurposing it, often leading to the specific harms of commodification and exploitation discussed in the next section.

\section{The Ethics of ``Taking'': Harm, Power, and Commodification}

Having defined the conditions of appropriation, we must ask: why is it morally wrong? \textcite{matthes_cultural_2019} provides the ``oppression account'' to answer this. He argues that the wrongness of cultural appropriation is not merely about hurt feelings or offense; rather, it is rooted in systemic imbalances of power. When a dominant culture takes from a marginalized one, the act is not neutral. It interacts with and exacerbates existing structures of oppression. In this view, appropriation is a political act that reinforces the marginalization of the source group.

A primary mechanism of this harm is ``commodification,'' which \textcite{arya_cultural_2021} describes as the process of turning a cultural good into an object of trade. In this transformation, the item's ``Use Value''---its spiritual, social, or community meaning---is stripped away and replaced by ``Exchange Value'', or its commercial worth. \textcite{schneider_appropriation_2003} refers to this as ``stripping context'', where a symbol is removed from its original web of meaning to serve a foreign purpose.

This process can lead to a phenomenon \textcite{arya_cultural_2021} calls the ``pizza effect'' or ``re-enculturation''. This occurs when a dominant culture appropriates a practice, repackages it, and sells it back to the originating culture, often displacing the original tradition. A pertinent example for UX design is the global wellness industry. Yoga and mindfulness, deeply rooted in Indian philosophy, have been repackaged by Western tech companies into sleek, gamified apps. These products are then marketed globally, including back to India, where the Westernized, commodified version often gains prestige over traditional practices. This cycle of harm reinforces cultural imperialism, where the dominant culture claims the power to define what is ``modern'' or ``valuable.''

However, this ethical stance is not universally accepted. \textcite{lenard_what_2020} note a ``sceptical view'' which argues that cultures have always borrowed from one another and that such exchange is natural. A more nuanced perspective is the ``Respectful Exchange View'' proposed by \textcite{young_cultural_2010}. Young argues that appropriation is not \textit{inherently} wrong if the appropriator provides proper acknowledgment and treats the material with respect. He warns that a strict prohibition on borrowing could stifle creativity. Taking this argument further, the novelist \textcite{shriver_lionel_2016} champions the ``Artistic Freedom View''. She dismisses the concept of appropriation as a ``passing fad'' and a form of ``surveillance'' that polices imagination, arguing that creators must have the right to ``wear many hats''.

While Shriver's argument may hold weight in the realm of fiction, it falters in the context of User Experience Design. Unlike pure art, UX is a functional, commercial discipline often driven by profit and utility. When a designer like Tom uses a religious icon to sell a subscription service, he is not engaging in a deep imaginative exercise; he is exploiting a marginalized culture's asset for financial gain. In this commercial context, the ``oppression account'' provided by Matthes is far more relevant than Shriver's defense of artistic license.

\section{Conclusion}
Summary of findings...

% --- Bibliography ---
\newpage
\pagenumbering{Roman} % Back matter uses Uppercase Roman (I, II...)
\printbibliography

% --- List of Figures & Glossary ---
\newpage
\listoffigures
\newpage
\printglossary[type=\acronymtype, title={Glossary}]

% --- Appendix ---
\newpage
\appendix
\section{Appendix Title}
Any supplementary material (interviews, large tables, etc.) goes here.

\end{document}
